%!TEX program = xelatex
%!TEX encoding = UTF-8 Unicode

\documentclass[14pt, AutoFakeBold]{ldr}
  

% 你要是非要把
\title{本 科 生 毕 业 答 辩}
\subtitle{毕业答辩题目题目题目题目题目题目题目题目}

% 记得改啊。。。。可不是我答辩 ~
\author{答辩人:余航}
\institute{指导老师:xx教授}
\date{\today}
\titlegraphic{\includegraphics[height=0.15\textwidth]{lzu_logo.png}}


\begin{document}

\maketitle

\AtBeginSection[]
{
  \begin{frame}
    \frametitle{\insertshorttitle}
    \tableofcontents[currentsection,hideallsubsections]
  \end{frame}
}

\section{研究背景}
\subsection{三级标题}


\begin{frame}[c]{列表怎么用}
就是下面这么用,前面是个点点,原因是我在模板里自定义了,你可以修改,模板中搜索:列表前面的点点自定义
  \begin{itemize}
  \item 简捷易行
  \item 保存原文的格调和“洋味”
  \item[1] 你是可以用数字的
  \item[2] 你是可以用数字的
  \end{itemize}
还有这一种,也挺好看的
  \begin{enumerate}
    \item 测试
    \item 测试
  \end{enumerate}
\end{frame}




\begin{frame}
  \frametitle{每一页内容位置}

  内容老是居中,我想让它居上怎么办?看下一页


\end{frame}

\begin{frame}[t]{每一页内容位置}

  这句话在上面了吧,原因是上面那个[t],默认[c](居中)


\end{frame}



\section{主要内容}

\begin{frame}
  \frametitle{图并排}
  \begin{figure}[H]
    \centering
    \includegraphics[width=0.3\textwidth]{lzu_logo.png}
  
    \caption{兰州大学(单图)\footnotesize 小字可以这么来}
    \label{fig_lzu}
\end{figure}
  
\end{frame}
\begin{frame}
  \frametitle{图并排}
  \begin{figure}[H]
    \centering
    \subfloat[图1]{
        \label{fig_lzus_0}
        \includegraphics[width=0.3\textwidth]{lzu_logo.png}
    }
    \subfloat[图2]{
        \label{fig_lzus_1}
        \includegraphics[width=0.3\textwidth]{lzu_logo.png}
    }\\
    \caption{兰州大学}
    \label{fig_lzus}
\end{figure}
  
\end{frame}


\begin{frame}
  \frametitle{表格}
  \begin{table}[H]
    \centering
    \caption{纳米管参数}
    \begin{tabular}{cccccc} % 控制表格的格式
        \toprule
        参数 & m  & n  & 原子数 & 内径     & 长度    \\
        \midrule
        二硫化钼纳米管  & 15 & 15 & 3420   & 2.3014nm & 11.85nm \\
        碳纳米管  & 16 & 6  & 1112   & 1.5424nm & 6.0nm   \\
        \bottomrule
    \end{tabular}
    \label{tbl_mcnt_nanotube}
\end{table}
\end{frame}

\begin{frame}
\frametitle{左右分栏}
去掉 pause  就不会变成两页了
  \begin{columns}
  \column{0.6\textwidth}

    \begin{itemize}
    \item Ice Age
    \item The Hobbit
    \item The Great Gatsby
    \end{itemize}
  \pause
  \column{0.4\textwidth}
    \begin{itemize}
    \item 冰河世纪
    \item 霍比特人
    \item 了不起的盖茨比
    \end{itemize}
  \end{columns}
\end{frame}

\begin{frame}
  \frametitle{分步动画}


  \onslide<1>{
    第一步显示这一句话,第二步时消失
  }
  \onslide<2,3>{
    \begin{block}{解析:}
      第二、三步显示这一句话
    \end{block}
  }

  \onslide<3>{
    只有第三步显示这一句话
  }

\end{frame}

\section{总结展望}

\begin{frame}
  \frametitle{测试}

  \begin{block}{方框}
    测试一下,参考一篇文献\cite{partl2016}
  \end{block}

\end{frame}


\section{参考文献}
\begin{frame}[t]
  \frametitle{参考文献}
  % 这里会把ppt引用的参考文献全部在这里显示
  \bibliographystyle{bib/lzubib}
  \bibliography{bib/database}
  
\end{frame}

\section{致谢}

\begin{frame}[t]
\frametitle{致谢}
  % 注意致谢内容答辩时不要读出来,播放到这里就可以了
  \begin{itemize}
  \item 感谢老师
  \item 感谢同学
  \item 感谢学校
  \item 感谢父母
  \item 也可以感谢我 ~
  \end{itemize}
\end{frame}






\end{document}